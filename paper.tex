\PassOptionsToPackage{unicode=true}{hyperref} % options for packages loaded elsewhere
\PassOptionsToPackage{hyphens}{url}
%
\documentclass[]{article}
\usepackage{lmodern}
\usepackage{amssymb,amsmath}
\usepackage{ifxetex,ifluatex}
\usepackage{fixltx2e} % provides \textsubscript
\ifnum 0\ifxetex 1\fi\ifluatex 1\fi=0 % if pdftex
  \usepackage[T1]{fontenc}
  \usepackage[utf8]{inputenc}
  \usepackage{textcomp} % provides euro and other symbols
\else % if luatex or xelatex
  \usepackage{unicode-math}
  \defaultfontfeatures{Ligatures=TeX,Scale=MatchLowercase}
\fi
% use upquote if available, for straight quotes in verbatim environments
\IfFileExists{upquote.sty}{\usepackage{upquote}}{}
% use microtype if available
\IfFileExists{microtype.sty}{%
\usepackage[]{microtype}
\UseMicrotypeSet[protrusion]{basicmath} % disable protrusion for tt fonts
}{}
\IfFileExists{parskip.sty}{%
\usepackage{parskip}
}{% else
\setlength{\parindent}{0pt}
\setlength{\parskip}{6pt plus 2pt minus 1pt}
}
\usepackage{hyperref}
\hypersetup{
            pdftitle={Final project},
            pdfauthor={Prayag Gordy},
            pdfborder={0 0 0},
            breaklinks=true}
\urlstyle{same}  % don't use monospace font for urls
\usepackage[margin=1in]{geometry}
\usepackage{longtable,booktabs}
% Fix footnotes in tables (requires footnote package)
\IfFileExists{footnote.sty}{\usepackage{footnote}\makesavenoteenv{longtable}}{}
\usepackage{graphicx,grffile}
\makeatletter
\def\maxwidth{\ifdim\Gin@nat@width>\linewidth\linewidth\else\Gin@nat@width\fi}
\def\maxheight{\ifdim\Gin@nat@height>\textheight\textheight\else\Gin@nat@height\fi}
\makeatother
% Scale images if necessary, so that they will not overflow the page
% margins by default, and it is still possible to overwrite the defaults
% using explicit options in \includegraphics[width, height, ...]{}
\setkeys{Gin}{width=\maxwidth,height=\maxheight,keepaspectratio}
\setlength{\emergencystretch}{3em}  % prevent overfull lines
\providecommand{\tightlist}{%
  \setlength{\itemsep}{0pt}\setlength{\parskip}{0pt}}
\setcounter{secnumdepth}{0}
% Redefines (sub)paragraphs to behave more like sections
\ifx\paragraph\undefined\else
\let\oldparagraph\paragraph
\renewcommand{\paragraph}[1]{\oldparagraph{#1}\mbox{}}
\fi
\ifx\subparagraph\undefined\else
\let\oldsubparagraph\subparagraph
\renewcommand{\subparagraph}[1]{\oldsubparagraph{#1}\mbox{}}
\fi

% set default figure placement to htbp
\makeatletter
\def\fps@figure{htbp}
\makeatother


\title{Final project}
\author{Prayag Gordy}
\date{5/4/2021}

\begin{document}
\maketitle

\hypertarget{project-statement}{%
\subsection{Project statement}\label{project-statement}}

Infant mortality is a widespread issue worldwide. Countless sums of
money are poured annually into research and programs to prevent the
death of children younger than one year. High rates of infant mortality,
which the United States Centers for Disease Control (CDC) defines as the
number of infant deaths per \(1,000\) live births, are common in South
Asia and Africa.

Targeted interventions and general improvements in access to and quality
of healthcare has decreased the infant mortality rate in the United
States, but disparities persist. People of color and of lower income are
at higher risk of losing a child before their first birthday. In this
project, I will examine infant mortality in 765 counties in the United
States. I will consider demographic information as predictors of
county-level infant mortality rates.

\hypertarget{data-description}{%
\subsection{Data description}\label{data-description}}

\begin{longtable}[]{@{}llll@{}}
\toprule
\begin{minipage}[b]{0.16\columnwidth}\raggedright
Data\strut
\end{minipage} & \begin{minipage}[b]{0.19\columnwidth}\raggedright
Source\strut
\end{minipage} & \begin{minipage}[b]{0.38\columnwidth}\raggedright
Link\strut
\end{minipage} & \begin{minipage}[b]{0.15\columnwidth}\raggedright
Collection method\strut
\end{minipage}\tabularnewline
\midrule
\endhead
\begin{minipage}[t]{0.16\columnwidth}\raggedright
Infant mortality data\strut
\end{minipage} & \begin{minipage}[t]{0.19\columnwidth}\raggedright
CDC WONDER\strut
\end{minipage} & \begin{minipage}[t]{0.38\columnwidth}\raggedright
\url{https://wonder.cdc.gov/controller/datarequest/D159}\strut
\end{minipage} & \begin{minipage}[t]{0.15\columnwidth}\raggedright
Download\strut
\end{minipage}\tabularnewline
\begin{minipage}[t]{0.16\columnwidth}\raggedright
Demographic data\strut
\end{minipage} & \begin{minipage}[t]{0.19\columnwidth}\raggedright
American Community Survey\strut
\end{minipage} & \begin{minipage}[t]{0.38\columnwidth}\raggedright
\url{https://www.census.gov/programs-surveys/acs}\strut
\end{minipage} & \begin{minipage}[t]{0.15\columnwidth}\raggedright
\texttt{tidycensus} package\strut
\end{minipage}\tabularnewline
\bottomrule
\end{longtable}

\begin{tabular}{}
\hline

\hline
\end{tabular}

The data were collected as described in Table @ref(fig:data-collection).
The indicators considered are listed in Table @ref(fig:indicators).

\hypertarget{exploratory-data-analysis}{%
\subsection{Exploratory data analysis}\label{exploratory-data-analysis}}

\hypertarget{data-analysis}{%
\subsection{Data analysis}\label{data-analysis}}

\hypertarget{summary-and-discussion}{%
\subsection{Summary and discussion}\label{summary-and-discussion}}

\hypertarget{references}{%
\subsection{References}\label{references}}

\end{document}
